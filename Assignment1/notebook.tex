
% Default to the notebook output style

    


% Inherit from the specified cell style.




    
\documentclass[11pt]{article}

    
    
    \usepackage[T1]{fontenc}
    % Nicer default font (+ math font) than Computer Modern for most use cases
    \usepackage{mathpazo}

    % Basic figure setup, for now with no caption control since it's done
    % automatically by Pandoc (which extracts ![](path) syntax from Markdown).
    \usepackage{graphicx}
    % We will generate all images so they have a width \maxwidth. This means
    % that they will get their normal width if they fit onto the page, but
    % are scaled down if they would overflow the margins.
    \makeatletter
    \def\maxwidth{\ifdim\Gin@nat@width>\linewidth\linewidth
    \else\Gin@nat@width\fi}
    \makeatother
    \let\Oldincludegraphics\includegraphics
    % Set max figure width to be 80% of text width, for now hardcoded.
    \renewcommand{\includegraphics}[1]{\Oldincludegraphics[width=.8\maxwidth]{#1}}
    % Ensure that by default, figures have no caption (until we provide a
    % proper Figure object with a Caption API and a way to capture that
    % in the conversion process - todo).
    \usepackage{caption}
    \DeclareCaptionLabelFormat{nolabel}{}
    \captionsetup{labelformat=nolabel}

    \usepackage{adjustbox} % Used to constrain images to a maximum size 
    \usepackage{xcolor} % Allow colors to be defined
    \usepackage{enumerate} % Needed for markdown enumerations to work
    \usepackage{geometry} % Used to adjust the document margins
    \usepackage{amsmath} % Equations
    \usepackage{amssymb} % Equations
    \usepackage{textcomp} % defines textquotesingle
    % Hack from http://tex.stackexchange.com/a/47451/13684:
    \AtBeginDocument{%
        \def\PYZsq{\textquotesingle}% Upright quotes in Pygmentized code
    }
    \usepackage{upquote} % Upright quotes for verbatim code
    \usepackage{eurosym} % defines \euro
    \usepackage[mathletters]{ucs} % Extended unicode (utf-8) support
    \usepackage[utf8x]{inputenc} % Allow utf-8 characters in the tex document
    \usepackage{fancyvrb} % verbatim replacement that allows latex
    \usepackage{grffile} % extends the file name processing of package graphics 
                         % to support a larger range 
    % The hyperref package gives us a pdf with properly built
    % internal navigation ('pdf bookmarks' for the table of contents,
    % internal cross-reference links, web links for URLs, etc.)
    \usepackage{hyperref}
    \usepackage{longtable} % longtable support required by pandoc >1.10
    \usepackage{booktabs}  % table support for pandoc > 1.12.2
    \usepackage[inline]{enumitem} % IRkernel/repr support (it uses the enumerate* environment)
    \usepackage[normalem]{ulem} % ulem is needed to support strikethroughs (\sout)
                                % normalem makes italics be italics, not underlines
    

    
    
    % Colors for the hyperref package
    \definecolor{urlcolor}{rgb}{0,.145,.698}
    \definecolor{linkcolor}{rgb}{.71,0.21,0.01}
    \definecolor{citecolor}{rgb}{.12,.54,.11}

    % ANSI colors
    \definecolor{ansi-black}{HTML}{3E424D}
    \definecolor{ansi-black-intense}{HTML}{282C36}
    \definecolor{ansi-red}{HTML}{E75C58}
    \definecolor{ansi-red-intense}{HTML}{B22B31}
    \definecolor{ansi-green}{HTML}{00A250}
    \definecolor{ansi-green-intense}{HTML}{007427}
    \definecolor{ansi-yellow}{HTML}{DDB62B}
    \definecolor{ansi-yellow-intense}{HTML}{B27D12}
    \definecolor{ansi-blue}{HTML}{208FFB}
    \definecolor{ansi-blue-intense}{HTML}{0065CA}
    \definecolor{ansi-magenta}{HTML}{D160C4}
    \definecolor{ansi-magenta-intense}{HTML}{A03196}
    \definecolor{ansi-cyan}{HTML}{60C6C8}
    \definecolor{ansi-cyan-intense}{HTML}{258F8F}
    \definecolor{ansi-white}{HTML}{C5C1B4}
    \definecolor{ansi-white-intense}{HTML}{A1A6B2}

    % commands and environments needed by pandoc snippets
    % extracted from the output of `pandoc -s`
    \providecommand{\tightlist}{%
      \setlength{\itemsep}{0pt}\setlength{\parskip}{0pt}}
    \DefineVerbatimEnvironment{Highlighting}{Verbatim}{commandchars=\\\{\}}
    % Add ',fontsize=\small' for more characters per line
    \newenvironment{Shaded}{}{}
    \newcommand{\KeywordTok}[1]{\textcolor[rgb]{0.00,0.44,0.13}{\textbf{{#1}}}}
    \newcommand{\DataTypeTok}[1]{\textcolor[rgb]{0.56,0.13,0.00}{{#1}}}
    \newcommand{\DecValTok}[1]{\textcolor[rgb]{0.25,0.63,0.44}{{#1}}}
    \newcommand{\BaseNTok}[1]{\textcolor[rgb]{0.25,0.63,0.44}{{#1}}}
    \newcommand{\FloatTok}[1]{\textcolor[rgb]{0.25,0.63,0.44}{{#1}}}
    \newcommand{\CharTok}[1]{\textcolor[rgb]{0.25,0.44,0.63}{{#1}}}
    \newcommand{\StringTok}[1]{\textcolor[rgb]{0.25,0.44,0.63}{{#1}}}
    \newcommand{\CommentTok}[1]{\textcolor[rgb]{0.38,0.63,0.69}{\textit{{#1}}}}
    \newcommand{\OtherTok}[1]{\textcolor[rgb]{0.00,0.44,0.13}{{#1}}}
    \newcommand{\AlertTok}[1]{\textcolor[rgb]{1.00,0.00,0.00}{\textbf{{#1}}}}
    \newcommand{\FunctionTok}[1]{\textcolor[rgb]{0.02,0.16,0.49}{{#1}}}
    \newcommand{\RegionMarkerTok}[1]{{#1}}
    \newcommand{\ErrorTok}[1]{\textcolor[rgb]{1.00,0.00,0.00}{\textbf{{#1}}}}
    \newcommand{\NormalTok}[1]{{#1}}
    
    % Additional commands for more recent versions of Pandoc
    \newcommand{\ConstantTok}[1]{\textcolor[rgb]{0.53,0.00,0.00}{{#1}}}
    \newcommand{\SpecialCharTok}[1]{\textcolor[rgb]{0.25,0.44,0.63}{{#1}}}
    \newcommand{\VerbatimStringTok}[1]{\textcolor[rgb]{0.25,0.44,0.63}{{#1}}}
    \newcommand{\SpecialStringTok}[1]{\textcolor[rgb]{0.73,0.40,0.53}{{#1}}}
    \newcommand{\ImportTok}[1]{{#1}}
    \newcommand{\DocumentationTok}[1]{\textcolor[rgb]{0.73,0.13,0.13}{\textit{{#1}}}}
    \newcommand{\AnnotationTok}[1]{\textcolor[rgb]{0.38,0.63,0.69}{\textbf{\textit{{#1}}}}}
    \newcommand{\CommentVarTok}[1]{\textcolor[rgb]{0.38,0.63,0.69}{\textbf{\textit{{#1}}}}}
    \newcommand{\VariableTok}[1]{\textcolor[rgb]{0.10,0.09,0.49}{{#1}}}
    \newcommand{\ControlFlowTok}[1]{\textcolor[rgb]{0.00,0.44,0.13}{\textbf{{#1}}}}
    \newcommand{\OperatorTok}[1]{\textcolor[rgb]{0.40,0.40,0.40}{{#1}}}
    \newcommand{\BuiltInTok}[1]{{#1}}
    \newcommand{\ExtensionTok}[1]{{#1}}
    \newcommand{\PreprocessorTok}[1]{\textcolor[rgb]{0.74,0.48,0.00}{{#1}}}
    \newcommand{\AttributeTok}[1]{\textcolor[rgb]{0.49,0.56,0.16}{{#1}}}
    \newcommand{\InformationTok}[1]{\textcolor[rgb]{0.38,0.63,0.69}{\textbf{\textit{{#1}}}}}
    \newcommand{\WarningTok}[1]{\textcolor[rgb]{0.38,0.63,0.69}{\textbf{\textit{{#1}}}}}
    
    
    % Define a nice break command that doesn't care if a line doesn't already
    % exist.
    \def\br{\hspace*{\fill} \\* }
    % Math Jax compatability definitions
    \def\gt{>}
    \def\lt{<}
    % Document parameters
    \title{kernel}
    
    
    

    % Pygments definitions
    
\makeatletter
\def\PY@reset{\let\PY@it=\relax \let\PY@bf=\relax%
    \let\PY@ul=\relax \let\PY@tc=\relax%
    \let\PY@bc=\relax \let\PY@ff=\relax}
\def\PY@tok#1{\csname PY@tok@#1\endcsname}
\def\PY@toks#1+{\ifx\relax#1\empty\else%
    \PY@tok{#1}\expandafter\PY@toks\fi}
\def\PY@do#1{\PY@bc{\PY@tc{\PY@ul{%
    \PY@it{\PY@bf{\PY@ff{#1}}}}}}}
\def\PY#1#2{\PY@reset\PY@toks#1+\relax+\PY@do{#2}}

\expandafter\def\csname PY@tok@w\endcsname{\def\PY@tc##1{\textcolor[rgb]{0.73,0.73,0.73}{##1}}}
\expandafter\def\csname PY@tok@c\endcsname{\let\PY@it=\textit\def\PY@tc##1{\textcolor[rgb]{0.25,0.50,0.50}{##1}}}
\expandafter\def\csname PY@tok@cp\endcsname{\def\PY@tc##1{\textcolor[rgb]{0.74,0.48,0.00}{##1}}}
\expandafter\def\csname PY@tok@k\endcsname{\let\PY@bf=\textbf\def\PY@tc##1{\textcolor[rgb]{0.00,0.50,0.00}{##1}}}
\expandafter\def\csname PY@tok@kp\endcsname{\def\PY@tc##1{\textcolor[rgb]{0.00,0.50,0.00}{##1}}}
\expandafter\def\csname PY@tok@kt\endcsname{\def\PY@tc##1{\textcolor[rgb]{0.69,0.00,0.25}{##1}}}
\expandafter\def\csname PY@tok@o\endcsname{\def\PY@tc##1{\textcolor[rgb]{0.40,0.40,0.40}{##1}}}
\expandafter\def\csname PY@tok@ow\endcsname{\let\PY@bf=\textbf\def\PY@tc##1{\textcolor[rgb]{0.67,0.13,1.00}{##1}}}
\expandafter\def\csname PY@tok@nb\endcsname{\def\PY@tc##1{\textcolor[rgb]{0.00,0.50,0.00}{##1}}}
\expandafter\def\csname PY@tok@nf\endcsname{\def\PY@tc##1{\textcolor[rgb]{0.00,0.00,1.00}{##1}}}
\expandafter\def\csname PY@tok@nc\endcsname{\let\PY@bf=\textbf\def\PY@tc##1{\textcolor[rgb]{0.00,0.00,1.00}{##1}}}
\expandafter\def\csname PY@tok@nn\endcsname{\let\PY@bf=\textbf\def\PY@tc##1{\textcolor[rgb]{0.00,0.00,1.00}{##1}}}
\expandafter\def\csname PY@tok@ne\endcsname{\let\PY@bf=\textbf\def\PY@tc##1{\textcolor[rgb]{0.82,0.25,0.23}{##1}}}
\expandafter\def\csname PY@tok@nv\endcsname{\def\PY@tc##1{\textcolor[rgb]{0.10,0.09,0.49}{##1}}}
\expandafter\def\csname PY@tok@no\endcsname{\def\PY@tc##1{\textcolor[rgb]{0.53,0.00,0.00}{##1}}}
\expandafter\def\csname PY@tok@nl\endcsname{\def\PY@tc##1{\textcolor[rgb]{0.63,0.63,0.00}{##1}}}
\expandafter\def\csname PY@tok@ni\endcsname{\let\PY@bf=\textbf\def\PY@tc##1{\textcolor[rgb]{0.60,0.60,0.60}{##1}}}
\expandafter\def\csname PY@tok@na\endcsname{\def\PY@tc##1{\textcolor[rgb]{0.49,0.56,0.16}{##1}}}
\expandafter\def\csname PY@tok@nt\endcsname{\let\PY@bf=\textbf\def\PY@tc##1{\textcolor[rgb]{0.00,0.50,0.00}{##1}}}
\expandafter\def\csname PY@tok@nd\endcsname{\def\PY@tc##1{\textcolor[rgb]{0.67,0.13,1.00}{##1}}}
\expandafter\def\csname PY@tok@s\endcsname{\def\PY@tc##1{\textcolor[rgb]{0.73,0.13,0.13}{##1}}}
\expandafter\def\csname PY@tok@sd\endcsname{\let\PY@it=\textit\def\PY@tc##1{\textcolor[rgb]{0.73,0.13,0.13}{##1}}}
\expandafter\def\csname PY@tok@si\endcsname{\let\PY@bf=\textbf\def\PY@tc##1{\textcolor[rgb]{0.73,0.40,0.53}{##1}}}
\expandafter\def\csname PY@tok@se\endcsname{\let\PY@bf=\textbf\def\PY@tc##1{\textcolor[rgb]{0.73,0.40,0.13}{##1}}}
\expandafter\def\csname PY@tok@sr\endcsname{\def\PY@tc##1{\textcolor[rgb]{0.73,0.40,0.53}{##1}}}
\expandafter\def\csname PY@tok@ss\endcsname{\def\PY@tc##1{\textcolor[rgb]{0.10,0.09,0.49}{##1}}}
\expandafter\def\csname PY@tok@sx\endcsname{\def\PY@tc##1{\textcolor[rgb]{0.00,0.50,0.00}{##1}}}
\expandafter\def\csname PY@tok@m\endcsname{\def\PY@tc##1{\textcolor[rgb]{0.40,0.40,0.40}{##1}}}
\expandafter\def\csname PY@tok@gh\endcsname{\let\PY@bf=\textbf\def\PY@tc##1{\textcolor[rgb]{0.00,0.00,0.50}{##1}}}
\expandafter\def\csname PY@tok@gu\endcsname{\let\PY@bf=\textbf\def\PY@tc##1{\textcolor[rgb]{0.50,0.00,0.50}{##1}}}
\expandafter\def\csname PY@tok@gd\endcsname{\def\PY@tc##1{\textcolor[rgb]{0.63,0.00,0.00}{##1}}}
\expandafter\def\csname PY@tok@gi\endcsname{\def\PY@tc##1{\textcolor[rgb]{0.00,0.63,0.00}{##1}}}
\expandafter\def\csname PY@tok@gr\endcsname{\def\PY@tc##1{\textcolor[rgb]{1.00,0.00,0.00}{##1}}}
\expandafter\def\csname PY@tok@ge\endcsname{\let\PY@it=\textit}
\expandafter\def\csname PY@tok@gs\endcsname{\let\PY@bf=\textbf}
\expandafter\def\csname PY@tok@gp\endcsname{\let\PY@bf=\textbf\def\PY@tc##1{\textcolor[rgb]{0.00,0.00,0.50}{##1}}}
\expandafter\def\csname PY@tok@go\endcsname{\def\PY@tc##1{\textcolor[rgb]{0.53,0.53,0.53}{##1}}}
\expandafter\def\csname PY@tok@gt\endcsname{\def\PY@tc##1{\textcolor[rgb]{0.00,0.27,0.87}{##1}}}
\expandafter\def\csname PY@tok@err\endcsname{\def\PY@bc##1{\setlength{\fboxsep}{0pt}\fcolorbox[rgb]{1.00,0.00,0.00}{1,1,1}{\strut ##1}}}
\expandafter\def\csname PY@tok@kc\endcsname{\let\PY@bf=\textbf\def\PY@tc##1{\textcolor[rgb]{0.00,0.50,0.00}{##1}}}
\expandafter\def\csname PY@tok@kd\endcsname{\let\PY@bf=\textbf\def\PY@tc##1{\textcolor[rgb]{0.00,0.50,0.00}{##1}}}
\expandafter\def\csname PY@tok@kn\endcsname{\let\PY@bf=\textbf\def\PY@tc##1{\textcolor[rgb]{0.00,0.50,0.00}{##1}}}
\expandafter\def\csname PY@tok@kr\endcsname{\let\PY@bf=\textbf\def\PY@tc##1{\textcolor[rgb]{0.00,0.50,0.00}{##1}}}
\expandafter\def\csname PY@tok@bp\endcsname{\def\PY@tc##1{\textcolor[rgb]{0.00,0.50,0.00}{##1}}}
\expandafter\def\csname PY@tok@fm\endcsname{\def\PY@tc##1{\textcolor[rgb]{0.00,0.00,1.00}{##1}}}
\expandafter\def\csname PY@tok@vc\endcsname{\def\PY@tc##1{\textcolor[rgb]{0.10,0.09,0.49}{##1}}}
\expandafter\def\csname PY@tok@vg\endcsname{\def\PY@tc##1{\textcolor[rgb]{0.10,0.09,0.49}{##1}}}
\expandafter\def\csname PY@tok@vi\endcsname{\def\PY@tc##1{\textcolor[rgb]{0.10,0.09,0.49}{##1}}}
\expandafter\def\csname PY@tok@vm\endcsname{\def\PY@tc##1{\textcolor[rgb]{0.10,0.09,0.49}{##1}}}
\expandafter\def\csname PY@tok@sa\endcsname{\def\PY@tc##1{\textcolor[rgb]{0.73,0.13,0.13}{##1}}}
\expandafter\def\csname PY@tok@sb\endcsname{\def\PY@tc##1{\textcolor[rgb]{0.73,0.13,0.13}{##1}}}
\expandafter\def\csname PY@tok@sc\endcsname{\def\PY@tc##1{\textcolor[rgb]{0.73,0.13,0.13}{##1}}}
\expandafter\def\csname PY@tok@dl\endcsname{\def\PY@tc##1{\textcolor[rgb]{0.73,0.13,0.13}{##1}}}
\expandafter\def\csname PY@tok@s2\endcsname{\def\PY@tc##1{\textcolor[rgb]{0.73,0.13,0.13}{##1}}}
\expandafter\def\csname PY@tok@sh\endcsname{\def\PY@tc##1{\textcolor[rgb]{0.73,0.13,0.13}{##1}}}
\expandafter\def\csname PY@tok@s1\endcsname{\def\PY@tc##1{\textcolor[rgb]{0.73,0.13,0.13}{##1}}}
\expandafter\def\csname PY@tok@mb\endcsname{\def\PY@tc##1{\textcolor[rgb]{0.40,0.40,0.40}{##1}}}
\expandafter\def\csname PY@tok@mf\endcsname{\def\PY@tc##1{\textcolor[rgb]{0.40,0.40,0.40}{##1}}}
\expandafter\def\csname PY@tok@mh\endcsname{\def\PY@tc##1{\textcolor[rgb]{0.40,0.40,0.40}{##1}}}
\expandafter\def\csname PY@tok@mi\endcsname{\def\PY@tc##1{\textcolor[rgb]{0.40,0.40,0.40}{##1}}}
\expandafter\def\csname PY@tok@il\endcsname{\def\PY@tc##1{\textcolor[rgb]{0.40,0.40,0.40}{##1}}}
\expandafter\def\csname PY@tok@mo\endcsname{\def\PY@tc##1{\textcolor[rgb]{0.40,0.40,0.40}{##1}}}
\expandafter\def\csname PY@tok@ch\endcsname{\let\PY@it=\textit\def\PY@tc##1{\textcolor[rgb]{0.25,0.50,0.50}{##1}}}
\expandafter\def\csname PY@tok@cm\endcsname{\let\PY@it=\textit\def\PY@tc##1{\textcolor[rgb]{0.25,0.50,0.50}{##1}}}
\expandafter\def\csname PY@tok@cpf\endcsname{\let\PY@it=\textit\def\PY@tc##1{\textcolor[rgb]{0.25,0.50,0.50}{##1}}}
\expandafter\def\csname PY@tok@c1\endcsname{\let\PY@it=\textit\def\PY@tc##1{\textcolor[rgb]{0.25,0.50,0.50}{##1}}}
\expandafter\def\csname PY@tok@cs\endcsname{\let\PY@it=\textit\def\PY@tc##1{\textcolor[rgb]{0.25,0.50,0.50}{##1}}}

\def\PYZbs{\char`\\}
\def\PYZus{\char`\_}
\def\PYZob{\char`\{}
\def\PYZcb{\char`\}}
\def\PYZca{\char`\^}
\def\PYZam{\char`\&}
\def\PYZlt{\char`\<}
\def\PYZgt{\char`\>}
\def\PYZsh{\char`\#}
\def\PYZpc{\char`\%}
\def\PYZdl{\char`\$}
\def\PYZhy{\char`\-}
\def\PYZsq{\char`\'}
\def\PYZdq{\char`\"}
\def\PYZti{\char`\~}
% for compatibility with earlier versions
\def\PYZat{@}
\def\PYZlb{[}
\def\PYZrb{]}
\makeatother


    % Exact colors from NB
    \definecolor{incolor}{rgb}{0.0, 0.0, 0.5}
    \definecolor{outcolor}{rgb}{0.545, 0.0, 0.0}



    
    % Prevent overflowing lines due to hard-to-break entities
    \sloppy 
    % Setup hyperref package
    \hypersetup{
      breaklinks=true,  % so long urls are correctly broken across lines
      colorlinks=true,
      urlcolor=urlcolor,
      linkcolor=linkcolor,
      citecolor=citecolor,
      }
    % Slightly bigger margins than the latex defaults
    
    \geometry{verbose,tmargin=1in,bmargin=1in,lmargin=1in,rmargin=1in}
    
    

    \begin{document}
    
    
    \maketitle
    
    

    
    \section{Analysis of Hurricane Data - A Case Study of Atlantic
Hurricanes}\label{analysis-of-hurricane-data---a-case-study-of-atlantic-hurricanes}

    \subsection{Loading the libraries}\label{loading-the-libraries}

    \begin{Verbatim}[commandchars=\\\{\}]
{\color{incolor}In [{\color{incolor}1}]:} \PY{c+c1}{\PYZsh{} For data munging and analysis}
        \PY{k+kn}{library}\PY{p}{(}tidyverse\PY{p}{)}
        
        \PY{c+c1}{\PYZsh{} For parsing the html tables from the internet}
        \PY{k+kn}{library}\PY{p}{(}rvest\PY{p}{)}
        
        \PY{c+c1}{\PYZsh{} Need to fit the data to various distributions}
        \PY{k+kn}{library}\PY{p}{(}fitdistrplus\PY{p}{)}
\end{Verbatim}


    \begin{Verbatim}[commandchars=\\\{\}]
── Attaching packages ─────────────────────────────────────── tidyverse 1.2.1 ──
✔ ggplot2 3.0.0.9000     ✔ purrr   0.2.5     
✔ tibble  1.4.2          ✔ dplyr   0.7.6     
✔ tidyr   0.8.1          ✔ stringr 1.3.1     
✔ readr   1.2.0          ✔ forcats 0.3.0     
── Conflicts ────────────────────────────────────────── tidyverse\_conflicts() ──
✖ dplyr::filter() masks stats::filter()
✖ dplyr::lag()    masks stats::lag()
Loading required package: xml2

Attaching package: ‘rvest’

The following object is masked from ‘package:purrr’:

    pluck

The following object is masked from ‘package:readr’:

    guess\_encoding

Loading required package: MASS

Attaching package: ‘MASS’

The following object is masked from ‘package:dplyr’:

    select

Loading required package: survival
Loading required package: npsurv
Loading required package: lsei

    \end{Verbatim}

    \subsection{Converting years to
decades}\label{converting-years-to-decades}

    \begin{Verbatim}[commandchars=\\\{\}]
{\color{incolor}In [{\color{incolor}2}]:} \PY{c+c1}{\PYZsh{} Generate a hurricane decade list to segment our hurricane data into the }
        \PY{c+c1}{\PYZsh{} corresponding decades}
        hurricane\PYZus{}decades \PY{o}{\PYZlt{}\PYZhy{}} \PY{k+kt}{c}\PY{p}{(}\PY{p}{)}
        hurricane\PYZus{}decade\PYZus{}year \PY{o}{\PYZlt{}\PYZhy{}} \PY{l+m}{1850}
        \PY{k+kr}{while} \PY{p}{(}hurricane\PYZus{}decade\PYZus{}year \PY{o}{\PYZlt{}=} \PY{l+m}{2020}\PY{p}{)}\PY{p}{\PYZob{}}
          hurricane\PYZus{}decades \PY{o}{=} \PY{k+kt}{c}\PY{p}{(}hurricane\PYZus{}decades\PY{p}{,} hurricane\PYZus{}decade\PYZus{}year\PY{p}{)}
          hurricane\PYZus{}decade\PYZus{}year \PY{o}{=} hurricane\PYZus{}decade\PYZus{}year \PY{o}{+} \PY{l+m}{10}
        \PY{p}{\PYZcb{}}
\end{Verbatim}


    \begin{Verbatim}[commandchars=\\\{\}]
{\color{incolor}In [{\color{incolor}3}]:} convert\PYZus{}seasons\PYZus{}to\PYZus{}decade \PY{o}{\PYZlt{}\PYZhy{}} \PY{k+kr}{function}\PY{p}{(}x\PY{p}{)}\PY{p}{\PYZob{}}
          index \PY{o}{\PYZlt{}\PYZhy{}} \PY{l+m}{1}
          \PY{k+kr}{while}\PY{p}{(}\PY{p}{(}index \PY{o}{+} \PY{l+m}{1}\PY{p}{)} \PY{o}{\PYZlt{}=} \PY{k+kp}{length}\PY{p}{(}hurricane\PYZus{}decades\PY{p}{)}\PY{p}{)}\PY{p}{\PYZob{}}
            
            \PY{k+kr}{if} \PY{p}{(}x \PY{o}{\PYZgt{}=} hurricane\PYZus{}decades\PY{p}{[[}index\PY{p}{]]} \PY{o}{\PYZam{}\PYZam{}} \PYZbs{}
                x\PY{o}{\PYZlt{}} hurricane\PYZus{}decades\PY{p}{[[}index \PY{o}{+} \PY{l+m}{1}\PY{p}{]]}\PY{p}{)}\PY{p}{\PYZob{}}
              \PY{k+kr}{return}\PY{p}{(}hurricane\PYZus{}decades\PY{p}{[[}index\PY{p}{]]}\PY{p}{)}
            \PY{p}{\PYZcb{}}
            index \PY{o}{\PYZlt{}\PYZhy{}} index \PY{o}{+} \PY{l+m}{1}
          \PY{p}{\PYZcb{}}
        \PY{p}{\PYZcb{}}
\end{Verbatim}


    \subsection{Parsing the html tables and process
them}\label{parsing-the-html-tables-and-process-them}

    \begin{Verbatim}[commandchars=\\\{\}]
{\color{incolor}In [{\color{incolor}4}]:} \PY{c+c1}{\PYZsh{} Function to get the html table given a url}
        get\PYZus{}html\PYZus{}table \PY{o}{\PYZlt{}\PYZhy{}} \PY{k+kr}{function}\PY{p}{(}\PY{k+kp}{url}\PY{p}{)}\PY{p}{\PYZob{}}
          hurricane\PYZus{}table \PY{o}{\PYZlt{}\PYZhy{}} url \PY{o}{\PYZpc{}\PYZgt{}\PYZpc{}} 
            read\PYZus{}html \PY{o}{\PYZpc{}\PYZgt{}\PYZpc{}}
            html\PYZus{}nodes\PY{p}{(}\PY{l+s}{\PYZdq{}}\PY{l+s}{table\PYZdq{}}\PY{p}{)}
          
          \PY{k+kr}{return}\PY{p}{(}hurricane\PYZus{}table\PY{p}{)}
        \PY{p}{\PYZcb{}}
\end{Verbatim}


    \begin{Verbatim}[commandchars=\\\{\}]
{\color{incolor}In [{\color{incolor}5}]:} \PY{c+c1}{\PYZsh{} Function to process category 4 hurricane tables}
        process\PYZus{}category\PYZus{}4\PYZus{}hurricane\PYZus{}tables \PY{o}{\PYZlt{}\PYZhy{}} \PY{k+kr}{function}\PY{p}{(}category\PYZus{}4\PYZus{}hurricane\PYZus{}tables\PY{p}{)}\PY{p}{\PYZob{}}
          processed\PYZus{}hurricane\PYZus{}tables \PY{o}{\PYZlt{}\PYZhy{}} \PY{k+kt}{c}\PY{p}{(}\PY{p}{)}
        
        \PY{c+c1}{\PYZsh{} There are 5 different tables on the page and we get all of them by }
        \PY{c+c1}{\PYZsh{} iterating through the raw hurricane table data from that page.}
          \PY{k+kr}{for} \PY{p}{(}index \PY{k+kr}{in} \PY{l+m}{2}\PY{o}{:}\PY{l+m}{6}\PY{p}{)}\PY{p}{\PYZob{}}
            
            \PY{c+c1}{\PYZsh{} First get the raw table}
            hurricane\PYZus{}table \PY{o}{\PYZlt{}\PYZhy{}} html\PYZus{}table\PY{p}{(}category\PYZus{}4\PYZus{}hurricane\PYZus{}tables\PY{p}{[}index\PY{p}{]}\PY{p}{,} fill \PY{o}{=} \PY{n+nb+bp}{T}\PY{p}{)}\PY{p}{[[}\PY{l+m}{1}\PY{p}{]]}
            
            \PY{c+c1}{\PYZsh{} The first row contains the column names, so we remove it}
            hurricane\PYZus{}table \PY{o}{\PYZlt{}\PYZhy{}} hurricane\PYZus{}table\PY{p}{[}\PY{l+m}{\PYZhy{}1}\PY{p}{,} \PY{p}{]}
            
            \PY{c+c1}{\PYZsh{} The last row also contains some unnecessary information, }
            \PY{c+c1}{\PYZsh{} so we remove the last row}
            hurricane\PYZus{}table \PY{o}{\PYZlt{}\PYZhy{}} \PY{k+kp}{head}\PY{p}{(}hurricane\PYZus{}table\PY{p}{,} \PY{l+m}{\PYZhy{}1}\PY{p}{)}
            
            \PY{c+c1}{\PYZsh{} Get only the Season column}
            hurricane\PYZus{}table \PY{o}{\PYZlt{}\PYZhy{}} hurricane\PYZus{}table\PY{p}{[}\PY{k+kt}{c}\PY{p}{(}\PY{l+s}{\PYZdq{}}\PY{l+s}{Season\PYZdq{}}\PY{p}{)}\PY{p}{]}
            
            \PY{c+c1}{\PYZsh{} Convert the Seasons to decades}
            hurricane\PYZus{}table\PY{o}{\PYZdl{}}Season\PYZus{}decade \PY{o}{\PYZlt{}\PYZhy{}} \PY{k+kp}{unlist}\PY{p}{(}\PY{k+kp}{lapply}\PY{p}{(}hurricane\PYZus{}table\PY{o}{\PYZdl{}}Season\PY{p}{,} 
                                                           convert\PYZus{}seasons\PYZus{}to\PYZus{}decade\PY{p}{)}\PY{p}{)}
            
            \PY{c+c1}{\PYZsh{} Group by Season\PYZus{}decade and sum the counts}
            hurricane\PYZus{}table\PY{o}{\PYZdl{}}one\PYZus{}column \PY{o}{=} \PY{l+m}{1}
            hurricane\PYZus{}table \PY{o}{\PYZlt{}\PYZhy{}} hurricane\PYZus{}table \PY{o}{\PYZpc{}\PYZgt{}\PYZpc{}}
              group\PYZus{}by\PY{p}{(}Season\PYZus{}decade\PY{p}{)} \PY{o}{\PYZpc{}\PYZgt{}\PYZpc{}}
              summarise\PY{p}{(}
                total\PYZus{}hurricanes \PY{o}{=} \PY{k+kp}{sum}\PY{p}{(}one\PYZus{}column\PY{p}{)}
              \PY{p}{)}
            
            \PY{c+c1}{\PYZsh{} Append the tables to the final list}
            processed\PYZus{}hurricane\PYZus{}tables\PY{p}{[[}index\PY{p}{]]} \PY{o}{\PYZlt{}\PYZhy{}} hurricane\PYZus{}table
          \PY{p}{\PYZcb{}}
          
          \PY{c+c1}{\PYZsh{} Different tables contain the row for the same decade and }
          \PY{c+c1}{\PYZsh{} hence we carry out a final group\PYZus{}by operation on the final processed}
          \PY{c+c1}{\PYZsh{} table to get the final hurricane counts.}
          processed\PYZus{}hurricane\PYZus{}tables \PY{o}{\PYZlt{}\PYZhy{}} bind\PYZus{}rows\PY{p}{(}processed\PYZus{}hurricane\PYZus{}tables\PY{p}{)}
          processed\PYZus{}hurricane\PYZus{}tables \PY{o}{\PYZlt{}\PYZhy{}} processed\PYZus{}hurricane\PYZus{}tables \PY{o}{\PYZpc{}\PYZgt{}\PYZpc{}}
              group\PYZus{}by\PY{p}{(}Season\PYZus{}decade\PY{p}{)} \PY{o}{\PYZpc{}\PYZgt{}\PYZpc{}}
              summarise\PY{p}{(}
                total\PYZus{}hurricanes \PY{o}{=} \PY{k+kp}{sum}\PY{p}{(}total\PYZus{}hurricanes\PY{p}{)}
              \PY{p}{)}
          \PY{k+kr}{return}\PY{p}{(}processed\PYZus{}hurricane\PYZus{}tables\PY{p}{)}
        \PY{p}{\PYZcb{}}
\end{Verbatim}


    \begin{Verbatim}[commandchars=\\\{\}]
{\color{incolor}In [{\color{incolor}6}]:} \PY{c+c1}{\PYZsh{} Function to process category 5 hurricane tables}
        process\PYZus{}category\PYZus{}5\PYZus{}hurricane\PYZus{}tables \PY{o}{\PYZlt{}\PYZhy{}} \PY{k+kr}{function}\PY{p}{(}category\PYZus{}5\PYZus{}hurricane\PYZus{}tables\PY{p}{)}\PY{p}{\PYZob{}}
          
          \PY{c+c1}{\PYZsh{} First get the raw table}
          hurricane\PYZus{}table \PY{o}{\PYZlt{}\PYZhy{}} html\PYZus{}table\PY{p}{(}category\PYZus{}5\PYZus{}hurricane\PYZus{}tables\PY{p}{[}\PY{l+m}{2}\PY{p}{]}\PY{p}{,} fill \PY{o}{=} \PY{n+nb+bp}{T}\PY{p}{)}\PY{p}{[[}\PY{l+m}{1}\PY{p}{]]}
          
          \PY{c+c1}{\PYZsh{} The first row contains the column names, so we remove it}
          hurricane\PYZus{}table \PY{o}{\PYZlt{}\PYZhy{}} hurricane\PYZus{}table\PY{p}{[}\PY{l+m}{\PYZhy{}1}\PY{p}{,} \PY{p}{]}
          
          \PY{c+c1}{\PYZsh{} The last row also contains some unnecessary information, }
          \PY{c+c1}{\PYZsh{} so we remove the last row}
          hurricane\PYZus{}table \PY{o}{\PYZlt{}\PYZhy{}} \PY{k+kp}{head}\PY{p}{(}hurricane\PYZus{}table\PY{p}{,} \PY{l+m}{\PYZhy{}1}\PY{p}{)}
          
          hurricane\PYZus{}table \PY{o}{\PYZlt{}\PYZhy{}} hurricane\PYZus{}table\PY{p}{[}\PY{k+kt}{c}\PY{p}{(}\PY{l+s}{\PYZdq{}}\PY{l+s}{Dates as aCategory 5\PYZdq{}}\PY{p}{)}\PY{p}{]}
          \PY{k+kp}{colnames}\PY{p}{(}hurricane\PYZus{}table\PY{p}{)} \PY{o}{\PYZlt{}\PYZhy{}} \PY{k+kt}{c}\PY{p}{(}\PY{l+s}{\PYZdq{}}\PY{l+s}{Dates\PYZdq{}}\PY{p}{)}
          
          \PY{c+c1}{\PYZsh{} Split the dates on space and get the 3rd element in the split }
          \PY{c+c1}{\PYZsh{} list because that is the year in the date.}
          hurricane\PYZus{}table\PY{o}{\PYZdl{}}Dates \PY{o}{\PYZlt{}\PYZhy{}} \PY{k+kp}{strsplit}\PY{p}{(}hurricane\PYZus{}table\PY{o}{\PYZdl{}}Dates\PY{p}{,} \PY{l+s}{\PYZsq{}}\PY{l+s}{ \PYZsq{}}\PY{p}{)}
          
          \PY{c+c1}{\PYZsh{} lapply returns a list and so we unlist/unpack the list to get the numbers}
          hurricane\PYZus{}table\PY{o}{\PYZdl{}}Season \PY{o}{\PYZlt{}\PYZhy{}} \PY{k+kp}{unlist}\PY{p}{(}\PY{k+kp}{lapply}\PY{p}{(}hurricane\PYZus{}table\PY{o}{\PYZdl{}}Dates\PY{p}{,} 
                                                  \PY{k+kr}{function}\PY{p}{(}x\PY{p}{)} x\PY{p}{[[}\PY{l+m}{3}\PY{p}{]]}\PY{p}{)}\PY{p}{)}
          
          \PY{c+c1}{\PYZsh{} Convert the Seasons to decades}
          hurricane\PYZus{}table\PY{o}{\PYZdl{}}Season\PYZus{}decade \PY{o}{\PYZlt{}\PYZhy{}} \PY{k+kp}{unlist}\PY{p}{(}\PY{k+kp}{lapply}\PY{p}{(}hurricane\PYZus{}table\PY{o}{\PYZdl{}}Season\PY{p}{,} 
                                                         convert\PYZus{}seasons\PYZus{}to\PYZus{}decade\PY{p}{)}\PY{p}{)}
          
          \PY{c+c1}{\PYZsh{} Group by Season\PYZus{}decade and sum the counts}
          hurricane\PYZus{}table\PY{o}{\PYZdl{}}one\PYZus{}column \PY{o}{=} \PY{l+m}{1}
          hurricane\PYZus{}table \PY{o}{\PYZlt{}\PYZhy{}} hurricane\PYZus{}table \PY{o}{\PYZpc{}\PYZgt{}\PYZpc{}}
            group\PYZus{}by\PY{p}{(}Season\PYZus{}decade\PY{p}{)} \PY{o}{\PYZpc{}\PYZgt{}\PYZpc{}}
            summarise\PY{p}{(}
              total\PYZus{}hurricanes \PY{o}{=} \PY{k+kp}{sum}\PY{p}{(}one\PYZus{}column\PY{p}{)}
            \PY{p}{)}
          
          \PY{k+kr}{return}\PY{p}{(}hurricane\PYZus{}table\PY{p}{)}
        \PY{p}{\PYZcb{}}
\end{Verbatim}


    \subsection{Get the html tables and convert to final processed
dataframes}\label{get-the-html-tables-and-convert-to-final-processed-dataframes}

    \begin{Verbatim}[commandchars=\\\{\}]
{\color{incolor}In [{\color{incolor}7}]:} \PY{c+c1}{\PYZsh{}URLs}
        wiki\PYZus{}url\PYZus{}for\PYZus{}category\PYZus{}4\PYZus{}hurricanes \PY{o}{\PYZlt{}\PYZhy{}} \PY{l+s}{\PYZdq{}}\PY{l+s}{https://en.wikipedia.org/wiki/List\PYZus{}of\PYZus{}Category\PYZus{}4\PYZus{}Atlantic\PYZus{}hurricanes\PYZdq{}}
        wiki\PYZus{}url\PYZus{}for\PYZus{}category\PYZus{}5\PYZus{}hurricanes \PY{o}{\PYZlt{}\PYZhy{}} \PY{l+s}{\PYZdq{}}\PY{l+s}{https://en.wikipedia.org/wiki/List\PYZus{}of\PYZus{}Category\PYZus{}5\PYZus{}Atlantic\PYZus{}hurricanes\PYZdq{}}
\end{Verbatim}


    \begin{Verbatim}[commandchars=\\\{\}]
{\color{incolor}In [{\color{incolor}8}]:} \PY{c+c1}{\PYZsh{} Fetching the Hurricane Tables}
        category\PYZus{}4\PYZus{}hurricane\PYZus{}tables \PY{o}{\PYZlt{}\PYZhy{}} get\PYZus{}html\PYZus{}table\PY{p}{(}wiki\PYZus{}url\PYZus{}for\PYZus{}category\PYZus{}4\PYZus{}hurricanes\PY{p}{)}
        category\PYZus{}5\PYZus{}hurricane\PYZus{}tables \PY{o}{\PYZlt{}\PYZhy{}} get\PYZus{}html\PYZus{}table\PY{p}{(}wiki\PYZus{}url\PYZus{}for\PYZus{}category\PYZus{}5\PYZus{}hurricanes\PY{p}{)}
\end{Verbatim}


    \begin{Verbatim}[commandchars=\\\{\}]
{\color{incolor}In [{\color{incolor}9}]:} \PY{c+c1}{\PYZsh{} Process the raw html tables and convert them to the structure needed}
        final\PYZus{}category\PYZus{}4\PYZus{}hurricane\PYZus{}table \PY{o}{\PYZlt{}\PYZhy{}} process\PYZus{}category\PYZus{}4\PYZus{}hurricane\PYZus{}tables\PY{p}{(}
            category\PYZus{}4\PYZus{}hurricane\PYZus{}tables\PY{p}{)}
        final\PYZus{}category\PYZus{}5\PYZus{}hurricane\PYZus{}table \PY{o}{\PYZlt{}\PYZhy{}} process\PYZus{}category\PYZus{}5\PYZus{}hurricane\PYZus{}tables\PY{p}{(}
            category\PYZus{}5\PYZus{}hurricane\PYZus{}tables\PY{p}{)}
        
        \PY{c+c1}{\PYZsh{} Combine both the tables}
        final\PYZus{}combined\PYZus{}hurricane\PYZus{}table \PY{o}{\PYZlt{}\PYZhy{}} bind\PYZus{}rows\PY{p}{(}final\PYZus{}category\PYZus{}4\PYZus{}hurricane\PYZus{}table\PY{p}{,} 
                                                    final\PYZus{}category\PYZus{}5\PYZus{}hurricane\PYZus{}table\PY{p}{)}
\end{Verbatim}


    \subsection{Analysing probability
distributions}\label{analysing-probability-distributions}

We will use the dpois() function of R to model the probability
distributions of our hurricane data. If we have the following code:

\textbf{dpois(x = 1, lambda = lambda\_for\_category\_4\_hurricanes)}

This is the \emph{Probability of 1 category-4 hurricane occuring in a
decade when the mean hurricane rate is equal to the mean category-4
hurricanes.}

Similarly, we can calculate the probability of \emph{n} category-4
hurricanes occuring in a decade.

    \begin{Verbatim}[commandchars=\\\{\}]
{\color{incolor}In [{\color{incolor}10}]:} lamba\PYZus{}for\PYZus{}category\PYZus{}4\PYZus{}hurricanes \PY{o}{\PYZlt{}\PYZhy{}} \PY{k+kp}{mean}\PY{p}{(}
             final\PYZus{}category\PYZus{}4\PYZus{}hurricane\PYZus{}table\PY{o}{\PYZdl{}}total\PYZus{}hurricanes\PY{p}{)}
         lamba\PYZus{}for\PYZus{}category\PYZus{}5\PYZus{}hurricanes \PY{o}{\PYZlt{}\PYZhy{}} \PY{k+kp}{mean}\PY{p}{(}
             final\PYZus{}category\PYZus{}5\PYZus{}hurricane\PYZus{}table\PY{o}{\PYZdl{}}total\PYZus{}hurricanes\PY{p}{)}
         lambda\PYZus{}for\PYZus{}combined\PYZus{}hurricanes \PY{o}{\PYZlt{}\PYZhy{}} \PY{k+kp}{mean}\PY{p}{(}
             final\PYZus{}combined\PYZus{}hurricane\PYZus{}table\PY{o}{\PYZdl{}}total\PYZus{}hurricanes\PY{p}{)}
\end{Verbatim}


    \begin{Verbatim}[commandchars=\\\{\}]
{\color{incolor}In [{\color{incolor}11}]:} get\PYZus{}poisson\PYZus{}probabilities \PY{o}{\PYZlt{}\PYZhy{}} \PY{k+kr}{function}\PY{p}{(}lambda\PY{p}{)}\PY{p}{\PYZob{}}
             
             poisson\PYZus{}probabilities \PY{o}{\PYZlt{}\PYZhy{}} \PY{k+kt}{c}\PY{p}{(}\PY{p}{)}
             \PY{k+kr}{for} \PY{p}{(}i \PY{k+kr}{in} \PY{l+m}{1}\PY{o}{:}\PY{l+m}{50}\PY{p}{)}\PY{p}{\PYZob{}}
                 
               \PY{c+c1}{\PYZsh{} Probability of exactly *i* hurricanes occuring in a decade}
               poisson\PYZus{}probabilities\PY{p}{[}i\PY{p}{]} \PY{o}{\PYZlt{}\PYZhy{}} dpois\PY{p}{(}i\PY{p}{,} lambda \PY{o}{=} \PY{l+m}{7}\PY{p}{)}
             \PY{p}{\PYZcb{}}
         
             poisson\PYZus{}probabilities\PYZus{}df \PY{o}{\PYZlt{}\PYZhy{}} \PY{k+kp}{as.data.frame}\PY{p}{(}poisson\PYZus{}probabilities\PY{p}{)}
             poisson\PYZus{}probabilities\PYZus{}df\PY{o}{\PYZdl{}}num\PYZus{}hurricanes\PYZus{}in\PYZus{}the\PYZus{}decade \PY{o}{\PYZlt{}\PYZhy{}} \PY{k+kt}{c}\PY{p}{(}\PY{l+m}{1}\PY{o}{:}\PY{l+m}{50}\PY{p}{)}
             \PY{k+kp}{colnames}\PY{p}{(}poisson\PYZus{}probabilities\PYZus{}df\PY{p}{)} \PY{o}{\PYZlt{}\PYZhy{}} \PY{k+kt}{c}\PY{p}{(}\PY{l+s}{\PYZdq{}}\PY{l+s}{poisson\PYZus{}probabilities\PYZdq{}}\PY{p}{,} 
                                                     \PY{l+s}{\PYZdq{}}\PY{l+s}{number\PYZus{}of\PYZus{}hurricanes\PYZdq{}}\PY{p}{)}
             
             \PY{k+kr}{return}\PY{p}{(}poisson\PYZus{}probabilities\PYZus{}df\PY{p}{)}
         \PY{p}{\PYZcb{}}
         
         category\PYZus{}4\PYZus{}poisson\PYZus{}probabilities\PYZus{}df \PY{o}{\PYZlt{}\PYZhy{}} get\PYZus{}poisson\PYZus{}probabilities\PY{p}{(}
             lambda \PY{o}{=} lamba\PYZus{}for\PYZus{}category\PYZus{}4\PYZus{}hurricanes\PY{p}{)}
         category\PYZus{}5\PYZus{}poisson\PYZus{}probabilities\PYZus{}df \PY{o}{\PYZlt{}\PYZhy{}} get\PYZus{}poisson\PYZus{}probabilities\PY{p}{(}
             lambda \PY{o}{=} lamba\PYZus{}for\PYZus{}category\PYZus{}5\PYZus{}hurricanes\PY{p}{)}
\end{Verbatim}


    \begin{Verbatim}[commandchars=\\\{\}]
{\color{incolor}In [{\color{incolor}12}]:} category\PYZus{}4\PYZus{}poisson\PYZus{}probabilities\PYZus{}df \PY{o}{\PYZpc{}\PYZgt{}\PYZpc{}}
         ggplot\PY{p}{(}aes\PY{p}{(}x \PY{o}{=} number\PYZus{}of\PYZus{}hurricanes\PY{p}{,} y \PY{o}{=} poisson\PYZus{}probabilities\PY{p}{)}\PY{p}{)}\PY{o}{+}
         geom\PYZus{}point\PY{p}{(}\PY{p}{)}\PY{o}{+}
         geom\PYZus{}line\PY{p}{(}color \PY{o}{=} \PY{l+s}{\PYZdq{}}\PY{l+s}{red\PYZdq{}}\PY{p}{)}\PY{o}{+}
         labs\PY{p}{(}x \PY{o}{=} \PY{l+s}{\PYZdq{}}\PY{l+s}{Number of Hurricanes in a Decade\PYZdq{}}\PY{p}{,} 
              y \PY{o}{=} \PY{l+s}{\PYZdq{}}\PY{l+s}{Probability of the Hurricane\PYZdq{}}\PY{p}{,} 
              title \PY{o}{=} \PY{l+s}{\PYZdq{}}\PY{l+s}{Probability Distribution of Number of Category \PYZhy{} 4 Hurricanes in a Decade\PYZdq{}}\PY{p}{)}
\end{Verbatim}


    
    
    \begin{center}
    \adjustimage{max size={0.9\linewidth}{0.9\paperheight}}{output_17_1.png}
    \end{center}
    { \hspace*{\fill} \\}
    
    \begin{Verbatim}[commandchars=\\\{\}]
{\color{incolor}In [{\color{incolor}13}]:} category\PYZus{}5\PYZus{}poisson\PYZus{}probabilities\PYZus{}df \PY{o}{\PYZpc{}\PYZgt{}\PYZpc{}}
         ggplot\PY{p}{(}aes\PY{p}{(}x \PY{o}{=} number\PYZus{}of\PYZus{}hurricanes\PY{p}{,} y \PY{o}{=} poisson\PYZus{}probabilities\PY{p}{)}\PY{p}{)}\PY{o}{+}
         geom\PYZus{}point\PY{p}{(}\PY{p}{)}\PY{o}{+}
         geom\PYZus{}line\PY{p}{(}color \PY{o}{=} \PY{l+s}{\PYZdq{}}\PY{l+s}{red\PYZdq{}}\PY{p}{)}\PY{o}{+}
         labs\PY{p}{(}x \PY{o}{=} \PY{l+s}{\PYZdq{}}\PY{l+s}{Number of Hurricanes in a Decade\PYZdq{}}\PY{p}{,} 
              y \PY{o}{=} \PY{l+s}{\PYZdq{}}\PY{l+s}{Probability of the Hurricane\PYZdq{}}\PY{p}{,} 
              title \PY{o}{=} \PY{l+s}{\PYZdq{}}\PY{l+s}{Probability Distribution of Number of Category \PYZhy{} 5 Hurricanes in a Decade\PYZdq{}}\PY{p}{)}
\end{Verbatim}


    
    
    \begin{center}
    \adjustimage{max size={0.9\linewidth}{0.9\paperheight}}{output_18_1.png}
    \end{center}
    { \hspace*{\fill} \\}
    
    We observe that the probability of a hurricane occuring (for both
categories), shows a distribution similar to a Poisson Distribution.

    \subsection{Analysing the cumulative probability
distribution}\label{analysing-the-cumulative-probability-distribution}

Let us look at another function:

\textbf{ppois(x = n, lambda = lambda\_for\_category\_4\_hurricanes)}

This function gives us the "Probability of \emph{n or fewer hurricanes}
occuring in a decade when the rate is the mean category-4 hurricanes".
Lets plot it

    \begin{Verbatim}[commandchars=\\\{\}]
{\color{incolor}In [{\color{incolor}14}]:} get\PYZus{}ppois\PYZus{}poisson\PYZus{}probabilities \PY{o}{\PYZlt{}\PYZhy{}} \PY{k+kr}{function}\PY{p}{(}lambda\PY{p}{)}\PY{p}{\PYZob{}}
             
             poisson\PYZus{}probabilities \PY{o}{\PYZlt{}\PYZhy{}} \PY{k+kt}{c}\PY{p}{(}\PY{p}{)}
             \PY{k+kr}{for} \PY{p}{(}i \PY{k+kr}{in} \PY{l+m}{1}\PY{o}{:}\PY{l+m}{50}\PY{p}{)}\PY{p}{\PYZob{}}
               \PY{c+c1}{\PYZsh{} Probability of *i* or less hurricanes occuring in a decade}
               poisson\PYZus{}probabilities\PY{p}{[}i\PY{p}{]} \PY{o}{\PYZlt{}\PYZhy{}} ppois\PY{p}{(}i\PY{p}{,} lambda \PY{o}{=} lambda\PY{p}{)}
             \PY{p}{\PYZcb{}}
         
             poisson\PYZus{}probabilities\PYZus{}df \PY{o}{\PYZlt{}\PYZhy{}} \PY{k+kp}{as.data.frame}\PY{p}{(}poisson\PYZus{}probabilities\PY{p}{)}
             poisson\PYZus{}probabilities\PYZus{}df\PY{o}{\PYZdl{}}num\PYZus{}hurricanes\PYZus{}in\PYZus{}the\PYZus{}decade \PY{o}{\PYZlt{}\PYZhy{}} \PY{k+kt}{c}\PY{p}{(}\PY{l+m}{1}\PY{o}{:}\PY{l+m}{50}\PY{p}{)}
             \PY{k+kp}{colnames}\PY{p}{(}poisson\PYZus{}probabilities\PYZus{}df\PY{p}{)} \PY{o}{\PYZlt{}\PYZhy{}} \PY{k+kt}{c}\PY{p}{(}\PY{l+s}{\PYZdq{}}\PY{l+s}{poisson\PYZus{}probabilities\PYZdq{}}\PY{p}{,} 
                                                     \PY{l+s}{\PYZdq{}}\PY{l+s}{number\PYZus{}of\PYZus{}hurricanes\PYZdq{}}\PY{p}{)}
             
             \PY{k+kr}{return}\PY{p}{(}poisson\PYZus{}probabilities\PYZus{}df\PY{p}{)}
         \PY{p}{\PYZcb{}}
         
         category\PYZus{}4\PYZus{}ppois\PYZus{}poisson\PYZus{}probabilities\PYZus{}df \PY{o}{\PYZlt{}\PYZhy{}} get\PYZus{}ppois\PYZus{}poisson\PYZus{}probabilities\PY{p}{(}
             lambda \PY{o}{=} lamba\PYZus{}for\PYZus{}category\PYZus{}4\PYZus{}hurricanes\PY{p}{)}
         category\PYZus{}5\PYZus{}ppois\PYZus{}poisson\PYZus{}probabilities\PYZus{}df \PY{o}{\PYZlt{}\PYZhy{}} get\PYZus{}ppois\PYZus{}poisson\PYZus{}probabilities\PY{p}{(}
             lambda \PY{o}{=} lamba\PYZus{}for\PYZus{}category\PYZus{}5\PYZus{}hurricanes\PY{p}{)}
\end{Verbatim}


    \begin{Verbatim}[commandchars=\\\{\}]
{\color{incolor}In [{\color{incolor}15}]:} category\PYZus{}4\PYZus{}ppois\PYZus{}poisson\PYZus{}probabilities\PYZus{}df \PY{o}{\PYZpc{}\PYZgt{}\PYZpc{}}
         ggplot\PY{p}{(}aes\PY{p}{(}x \PY{o}{=} number\PYZus{}of\PYZus{}hurricanes\PY{p}{,} y \PY{o}{=} poisson\PYZus{}probabilities\PY{p}{)}\PY{p}{)}\PY{o}{+}
         geom\PYZus{}point\PY{p}{(}\PY{p}{)}\PY{o}{+}
         geom\PYZus{}line\PY{p}{(}color \PY{o}{=} \PY{l+s}{\PYZdq{}}\PY{l+s}{red\PYZdq{}}\PY{p}{)}\PY{o}{+}
         labs\PY{p}{(}x \PY{o}{=} \PY{l+s}{\PYZdq{}}\PY{l+s}{Number of Hurricanes in a Decade\PYZdq{}}\PY{p}{,} 
              y \PY{o}{=} \PY{l+s}{\PYZdq{}}\PY{l+s}{Probability of the Hurricane\PYZdq{}}\PY{p}{,} 
              title \PY{o}{=} \PY{l+s}{\PYZdq{}}\PY{l+s}{Probability Distribution of n or fewer Category \PYZhy{} 4 Hurricanes in a Decade\PYZdq{}}\PY{p}{)}
\end{Verbatim}


    
    
    \begin{center}
    \adjustimage{max size={0.9\linewidth}{0.9\paperheight}}{output_22_1.png}
    \end{center}
    { \hspace*{\fill} \\}
    
    \begin{Verbatim}[commandchars=\\\{\}]
{\color{incolor}In [{\color{incolor}16}]:} category\PYZus{}5\PYZus{}ppois\PYZus{}poisson\PYZus{}probabilities\PYZus{}df \PY{o}{\PYZpc{}\PYZgt{}\PYZpc{}}
         ggplot\PY{p}{(}aes\PY{p}{(}x \PY{o}{=} number\PYZus{}of\PYZus{}hurricanes\PY{p}{,} y \PY{o}{=} poisson\PYZus{}probabilities\PY{p}{)}\PY{p}{)}\PY{o}{+}
         geom\PYZus{}point\PY{p}{(}\PY{p}{)}\PY{o}{+}
         geom\PYZus{}line\PY{p}{(}color \PY{o}{=} \PY{l+s}{\PYZdq{}}\PY{l+s}{red\PYZdq{}}\PY{p}{)}\PY{o}{+}
         labs\PY{p}{(}x \PY{o}{=} \PY{l+s}{\PYZdq{}}\PY{l+s}{Number of Hurricanes in a Decade\PYZdq{}}\PY{p}{,} 
              y \PY{o}{=} \PY{l+s}{\PYZdq{}}\PY{l+s}{Probability of the Hurricane\PYZdq{}}\PY{p}{,} 
              title \PY{o}{=} \PY{l+s}{\PYZdq{}}\PY{l+s}{Probability Distribution of n or fewer Category \PYZhy{} 5 Hurricanes in a Decade\PYZdq{}}\PY{p}{)}
\end{Verbatim}


    
    
    \begin{center}
    \adjustimage{max size={0.9\linewidth}{0.9\paperheight}}{output_23_1.png}
    \end{center}
    { \hspace*{\fill} \\}
    
    \subsection{Is it an exact Poisson
distribution?}\label{is-it-an-exact-poisson-distribution}

    Although the plots look similar to a Poisson distribution, we cannot
assume that the data is indeed a Poisson distribution by looking at the
plots. Poisson distribution has a property that the mean and variance
are equal and we use this property to test the fit of our data. Wecheck
if this property is satisfied or \emph{almost satisfied} by our data. We
compute the ratio of mean and variance of our data:

    \begin{Verbatim}[commandchars=\\\{\}]
{\color{incolor}In [{\color{incolor}17}]:} variance\PYZus{}of\PYZus{}cat\PYZus{}4\PYZus{}hurricanes \PY{o}{\PYZlt{}\PYZhy{}} var\PY{p}{(}
             final\PYZus{}category\PYZus{}4\PYZus{}hurricane\PYZus{}table\PY{o}{\PYZdl{}}total\PYZus{}hurricanes\PY{p}{)}
         mean\PYZus{}of\PYZus{}cat\PYZus{}4\PYZus{}hurricanes \PY{o}{\PYZlt{}\PYZhy{}} \PY{k+kp}{mean}\PY{p}{(}
             final\PYZus{}category\PYZus{}4\PYZus{}hurricane\PYZus{}table\PY{o}{\PYZdl{}}total\PYZus{}hurricanes\PY{p}{)}
         
         variance\PYZus{}mean\PYZus{}ratio\PYZus{}for\PYZus{}category\PYZus{}4\PYZus{}hurricanes \PY{o}{\PYZlt{}\PYZhy{}} \PY{k+kp}{round}\PY{p}{(}
             variance\PYZus{}of\PYZus{}cat\PYZus{}4\PYZus{}hurricanes\PY{o}{/}mean\PYZus{}of\PYZus{}cat\PYZus{}4\PYZus{}hurricanes\PY{p}{,} \PY{l+m}{2}\PY{p}{)}
         variance\PYZus{}mean\PYZus{}ratio\PYZus{}for\PYZus{}category\PYZus{}4\PYZus{}hurricanes
\end{Verbatim}


    2.82

    
    We see that the variance of our data is 2.82 times larger than the mean
of the data. We check if such a behaviour is normal or not for a poisson
distribution. We check if this is just a one time anomaly or is it
recurring or not. We will perform a \textbf{Monte Carlo Simulation} and
repeatedly sample values from the poisson distribution with the
mean/lambda equal to the mean of the hurricane data. Our main aim of the
simulation is to check -

\emph{Assuming that the category 4 hurricane data is a perfect Poisson
distribution, how likely it is for us to generate samples with a
variance-to-mean ratio equal to 2.82}

    \begin{Verbatim}[commandchars=\\\{\}]
{\color{incolor}In [{\color{incolor}18}]:} \PY{k+kp}{set.seed}\PY{p}{(}\PY{l+m}{420}\PY{p}{)}
         variance\PYZus{}mean\PYZus{}ratio\PYZus{}from\PYZus{}samples \PY{o}{\PYZlt{}\PYZhy{}} \PY{k+kt}{c}\PY{p}{(}\PY{p}{)}
         num\PYZus{}of\PYZus{}monte\PYZus{}carlo\PYZus{}simulations \PY{o}{\PYZlt{}\PYZhy{}} \PY{l+m}{1000}
         num\PYZus{}of\PYZus{}points\PYZus{}in\PYZus{}sample \PY{o}{\PYZlt{}\PYZhy{}} \PY{k+kp}{nrow}\PY{p}{(}final\PYZus{}category\PYZus{}4\PYZus{}hurricane\PYZus{}table\PY{p}{)}
         
         \PY{k+kr}{for} \PY{p}{(}monte\PYZus{}carlo\PYZus{}experiment\PYZus{}index \PY{k+kr}{in}  \PY{l+m}{1}\PY{o}{:}num\PYZus{}of\PYZus{}monte\PYZus{}carlo\PYZus{}simulations\PY{p}{)}\PY{p}{\PYZob{}}
             poisson\PYZus{}sample \PY{o}{\PYZlt{}\PYZhy{}} rpois\PY{p}{(}n \PY{o}{=} num\PYZus{}of\PYZus{}points\PYZus{}in\PYZus{}sample\PY{p}{,} 
                                     lambda \PY{o}{=} lamba\PYZus{}for\PYZus{}category\PYZus{}4\PYZus{}hurricanes\PY{p}{)}
             variance\PYZus{}mean\PYZus{}ratio\PYZus{}from\PYZus{}samples\PY{p}{[[}monte\PYZus{}carlo\PYZus{}experiment\PYZus{}index\PY{p}{]]} \PY{o}{\PYZlt{}\PYZhy{}} var\PY{p}{(}
                 poisson\PYZus{}sample\PY{p}{)}\PY{o}{/}\PY{k+kp}{mean}\PY{p}{(}poisson\PYZus{}sample\PY{p}{)}
         \PY{p}{\PYZcb{}}
         
         percentage\PYZus{}of\PYZus{}samples\PYZus{}with\PYZus{}variance\PYZus{}mean\PYZus{}ratio\PYZus{}greater\PYZus{}than\PYZus{}2.82 \PY{o}{\PYZlt{}\PYZhy{}} \PY{k+kp}{sum}\PY{p}{(}
             variance\PYZus{}mean\PYZus{}ratio\PYZus{}from\PYZus{}samples \PY{o}{\PYZgt{}} \PYZbs{}
             variance\PYZus{}mean\PYZus{}ratio\PYZus{}for\PYZus{}category\PYZus{}4\PYZus{}hurricanes\PY{p}{)}\PY{o}{*}\PY{l+m}{100}\PY{o}{/}num\PYZus{}of\PYZus{}monte\PYZus{}carlo\PYZus{}simulations
         percentage\PYZus{}of\PYZus{}samples\PYZus{}with\PYZus{}variance\PYZus{}mean\PYZus{}ratio\PYZus{}greater\PYZus{}than\PYZus{}2.82
\end{Verbatim}


    0.1

    
    This shows that only 0.1\% of the monte carlo samples have a
variance-mean ratio greater than 2.82. Hence, the hurricane count in a
decade does not exhibit an exact Poisson process and the variability in
hurricane counts is higher than a Poisson distribution with constant
rate. This means that for a distribution of hurricane counts in a
decade, the lambda/rate is not constant but keeps changing.

    \subsection{Reasons for the varying lambda/rate in hurricane
data}\label{reasons-for-the-varying-lambdarate-in-hurricane-data}

    The reasons for the non-constant rate/lambda in our hurricane data is
because external climatic conditions affect the occurence of hurricane
and ultimately change the lambda. These external factors could be
\textbf{changes in pressure, wind speeds, El Nino etc...} This leads to
hurricane data being a varying Poisson distribution or an
\textbf{inhomogeneous Poisson distribution} which can be described as a
Poisson distribution with a variable rate/lambda.

    \subsection{Analysing QQ Plot}\label{analysing-qq-plot}

Quantile plots are a good way to look at what distribution a data might
belong to. Here, we plot a quantile plot of our hurricane data (using
the category-4 data again) and quantiles drawn from a theoretical
Poisson distribution.

    \begin{Verbatim}[commandchars=\\\{\}]
{\color{incolor}In [{\color{incolor}19}]:} qqcomp\PY{p}{(}fitdist\PY{p}{(}final\PYZus{}category\PYZus{}4\PYZus{}hurricane\PYZus{}table\PY{o}{\PYZdl{}}total\PYZus{}hurricanes\PY{p}{,} \PY{l+s}{\PYZdq{}}\PY{l+s}{pois\PYZdq{}}\PY{p}{)}\PY{p}{)}
\end{Verbatim}


    \begin{center}
    \adjustimage{max size={0.9\linewidth}{0.9\paperheight}}{output_33_0.png}
    \end{center}
    { \hspace*{\fill} \\}
    
    The quantile plot strengthens our conclusion that our hurricane data is
not entirely a Poisson distribution with a constant rate. What if we use
a negative binomial distribution for this?

    \begin{Verbatim}[commandchars=\\\{\}]
{\color{incolor}In [{\color{incolor}20}]:} qqcomp\PY{p}{(}fitdist\PY{p}{(}final\PYZus{}category\PYZus{}4\PYZus{}hurricane\PYZus{}table\PY{o}{\PYZdl{}}total\PYZus{}hurricanes\PY{p}{,} \PY{l+s}{\PYZdq{}}\PY{l+s}{nbinom\PYZdq{}}\PY{p}{)}\PY{p}{)}
\end{Verbatim}


    \begin{center}
    \adjustimage{max size={0.9\linewidth}{0.9\paperheight}}{output_35_0.png}
    \end{center}
    { \hspace*{\fill} \\}
    
    This shows that the hurricane count data infact is similar to a
\textbf{negative binomial distribution}. Poisson distributions are
special cases of negative binomial distributions and our above
distribution is a case of \textbf{overdispersed Poisson distribution}.
In an overdispersed Poisson distribution, the observations are
overdispersed in comparison to a theoretical Poisson distribution where
variance is equal to the mean. This overdispersion causes the variance
of the data to be greater than the mean - which is the case for our
hurricane data. Such overdispersion can be reduced by varying the
variance and keeping the mean constant. Since negative binomial
distributions have one more parameter than a Poisson distribution, we
can vary the parameter to adjust the variance keeping the mean constant.

\emph{References for the above explanation - }

{[}1{]}
https://stats.stackexchange.com/questions/32035/checking-poisson-distribution-plot-using-mean-and-variance-relationship

{[}2{]}
https://en.wikipedia.org/wiki/Negative\_binomial\_distribution\#Overdispersed\_Poisson

    \subsection{Analysing the variables - an exploratory data
analysis}\label{analysing-the-variables---an-exploratory-data-analysis}

Having

    \begin{Verbatim}[commandchars=\\\{\}]
{\color{incolor}In [{\color{incolor}21}]:} final\PYZus{}category\PYZus{}4\PYZus{}hurricane\PYZus{}table \PY{o}{\PYZpc{}\PYZgt{}\PYZpc{}}
         ggplot\PY{p}{(}aes\PY{p}{(}\PY{k+kp}{as.factor}\PY{p}{(}Season\PYZus{}decade\PY{p}{)}\PY{p}{,} total\PYZus{}hurricanes\PY{p}{)}\PY{p}{)}\PY{o}{+}
         geom\PYZus{}bar\PY{p}{(}stat \PY{o}{=} \PY{l+s}{\PYZdq{}}\PY{l+s}{identity\PYZdq{}}\PY{p}{,} color \PY{o}{=} \PY{l+s}{\PYZdq{}}\PY{l+s}{black\PYZdq{}}\PY{p}{,} fill \PY{o}{=} \PY{l+s}{\PYZdq{}}\PY{l+s}{tomato\PYZdq{}}\PY{p}{)}\PY{o}{+}
         geom\PYZus{}line\PY{p}{(}\PY{p}{)}\PY{o}{+}
         labs\PY{p}{(}x \PY{o}{=} \PY{l+s}{\PYZdq{}}\PY{l+s}{Decades\PYZdq{}}\PY{p}{,} 
              y \PY{o}{=} \PY{l+s}{\PYZdq{}}\PY{l+s}{Number of Hurricanes\PYZdq{}}\PY{p}{,} 
              title \PY{o}{=} \PY{l+s}{\PYZdq{}}\PY{l+s}{Count of Category \PYZhy{} 4 Hurricanes over the Decades\PYZdq{}}\PY{p}{)}
\end{Verbatim}


    \begin{Verbatim}[commandchars=\\\{\}]
geom\_path: Each group consists of only one observation. Do you need to adjust
the group aesthetic?

    \end{Verbatim}

    
    
    \begin{center}
    \adjustimage{max size={0.9\linewidth}{0.9\paperheight}}{output_38_2.png}
    \end{center}
    { \hspace*{\fill} \\}
    
    We see that the number of category-4 hurricanes occuring in a decade has
increased with time gradually.

    \begin{Verbatim}[commandchars=\\\{\}]
{\color{incolor}In [{\color{incolor}22}]:} final\PYZus{}category\PYZus{}5\PYZus{}hurricane\PYZus{}table \PY{o}{\PYZpc{}\PYZgt{}\PYZpc{}}
         ggplot\PY{p}{(}aes\PY{p}{(}\PY{k+kp}{as.factor}\PY{p}{(}Season\PYZus{}decade\PY{p}{)}\PY{p}{,} total\PYZus{}hurricanes\PY{p}{)}\PY{p}{)}\PY{o}{+}
         geom\PYZus{}bar\PY{p}{(}stat \PY{o}{=} \PY{l+s}{\PYZdq{}}\PY{l+s}{identity\PYZdq{}}\PY{p}{,} color \PY{o}{=} \PY{l+s}{\PYZdq{}}\PY{l+s}{black\PYZdq{}}\PY{p}{,} fill \PY{o}{=} \PY{l+s}{\PYZdq{}}\PY{l+s}{skyblue\PYZdq{}}\PY{p}{)}\PY{o}{+}
         labs\PY{p}{(}x \PY{o}{=} \PY{l+s}{\PYZdq{}}\PY{l+s}{Decades\PYZdq{}}\PY{p}{,} 
              y \PY{o}{=} \PY{l+s}{\PYZdq{}}\PY{l+s}{Number of Hurricanes\PYZdq{}}\PY{p}{,} 
              title \PY{o}{=} \PY{l+s}{\PYZdq{}}\PY{l+s}{Count of Category \PYZhy{} 5 Hurricanes over the Decades\PYZdq{}}\PY{p}{)}
\end{Verbatim}


    
    
    \begin{center}
    \adjustimage{max size={0.9\linewidth}{0.9\paperheight}}{output_40_1.png}
    \end{center}
    { \hspace*{\fill} \\}
    
    However, in the case of category-5 hurricanes, the distribution/count of
hurricanes is not

    Now, let us upload the whole raw data and clean it again to draw some
more analysis. (Till now, the dataset we were working on was decade
level)

    \begin{Verbatim}[commandchars=\\\{\}]
{\color{incolor}In [{\color{incolor}23}]:} \PY{c+c1}{\PYZsh{} Load whole of csv data and not just decade wise}
         category\PYZus{}4 \PY{o}{=} read.csv\PY{p}{(}\PY{l+s}{\PYZdq{}}\PY{l+s}{../input/category\PYZhy{}4/Category4Hurricanes \PYZhy{} Sheet1.csv\PYZdq{}}\PY{p}{,} 
                               header \PY{o}{=} \PY{k+kc}{TRUE}\PY{p}{)}
         category\PYZus{}5 \PY{o}{=} read.csv\PY{p}{(}\PY{l+s}{\PYZdq{}}\PY{l+s}{../input/category\PYZhy{}5/Category5Hurricanes \PYZhy{} Sheet1.csv\PYZdq{}}\PY{p}{,} 
                               header \PY{o}{=} \PY{k+kc}{TRUE}\PY{p}{)}
\end{Verbatim}


    \begin{Verbatim}[commandchars=\\\{\}]
{\color{incolor}In [{\color{incolor}24}]:} \PY{c+c1}{\PYZsh{} Changing column names for cateogry\PYZus{}4 dataset}
         \PY{k+kp}{colnames}\PY{p}{(}category\PYZus{}4\PY{p}{)} \PY{o}{\PYZlt{}\PYZhy{}} \PY{k+kt}{c}\PY{p}{(}\PY{l+s}{\PYZdq{}}\PY{l+s}{Name\PYZdq{}}\PY{p}{,} \PY{l+s}{\PYZdq{}}\PY{l+s}{Season\PYZdq{}}\PY{p}{,} \PY{l+s}{\PYZdq{}}\PY{l+s}{Month\PYZdq{}}\PY{p}{,} \PY{l+s}{\PYZdq{}}\PY{l+s}{Max\PYZus{}wind\PYZus{}knots\PYZdq{}}\PY{p}{,} 
                                   \PY{l+s}{\PYZdq{}}\PY{l+s}{Max\PYZus{}wind\PYZus{}kmh\PYZdq{}}\PY{p}{,} \PY{l+s}{\PYZdq{}}\PY{l+s}{Max\PYZus{}wind\PYZus{}mph\PYZdq{}}\PY{p}{,} \PY{l+s}{\PYZdq{}}\PY{l+s}{Min\PYZus{}pressure\PYZus{}mbar\PYZdq{}}\PY{p}{)}
         
         \PY{c+c1}{\PYZsh{} Changing column names for cateogry\PYZus{}5 dataset}
         \PY{k+kp}{colnames}\PY{p}{(}category\PYZus{}5\PY{p}{)} \PY{o}{\PYZlt{}\PYZhy{}} \PY{k+kt}{c}\PY{p}{(}\PY{l+s}{\PYZdq{}}\PY{l+s}{Name\PYZdq{}}\PY{p}{,} \PY{l+s}{\PYZdq{}}\PY{l+s}{Dates\PYZdq{}}\PY{p}{,} \PY{l+s}{\PYZdq{}}\PY{l+s}{Duration\PYZus{}hours\PYZdq{}}\PY{p}{,} \PY{l+s}{\PYZdq{}}\PY{l+s}{WindSpeedsMPH\PYZdq{}}\PY{p}{,} 
                                   \PY{l+s}{\PYZdq{}}\PY{l+s}{PressurehPA\PYZdq{}}\PY{p}{,} \PY{l+s}{\PYZdq{}}\PY{l+s}{Affected\PYZus{}Areas\PYZdq{}}\PY{p}{,} \PY{l+s}{\PYZdq{}}\PY{l+s}{Deaths\PYZdq{}}\PY{p}{,} 
                                   \PY{l+s}{\PYZdq{}}\PY{l+s}{DamageUSDMillions\PYZdq{}}\PY{p}{)}
         
         \PY{c+c1}{\PYZsh{} Clean month column for Category 4}
         category\PYZus{}4\PY{o}{\PYZdl{}}Month \PY{o}{\PYZlt{}\PYZhy{}} \PY{k+kp}{gsub}\PY{p}{(}\PY{l+s}{\PYZdq{}}\PY{l+s}{ .*\PYZdl{}\PYZdq{}}\PY{p}{,} \PY{l+s}{\PYZdq{}}\PY{l+s}{\PYZdq{}}\PY{p}{,}category\PYZus{}4\PY{o}{\PYZdl{}}Month\PY{p}{)}
         category\PYZus{}4\PY{o}{\PYZdl{}}Month \PY{o}{\PYZlt{}\PYZhy{}} \PY{k+kp}{gsub}\PY{p}{(}\PY{l+s}{\PYZdq{}}\PY{l+s}{,\PYZdq{}}\PY{p}{,} \PY{l+s}{\PYZdq{}}\PY{l+s}{\PYZdq{}}\PY{p}{,} category\PYZus{}4\PY{o}{\PYZdl{}}Month\PY{p}{)}
         
         \PY{c+c1}{\PYZsh{} Clean min\PYZus{}pressure column for Category 4}
         from \PY{o}{\PYZlt{}\PYZhy{}} \PY{k+kt}{c}\PY{p}{(}\PY{l+s}{\PYZdq{}}\PY{l+s}{≤ \PYZdq{}}\PY{p}{,}\PY{l+s}{\PYZdq{}}\PY{l+s}{\PYZhy{}\PYZdq{}}\PY{p}{,} \PY{l+s}{\PYZdq{}}\PY{l+s}{–\PYZdq{}}\PY{p}{)}
         to \PY{o}{\PYZlt{}\PYZhy{}} \PY{k+kt}{c}\PY{p}{(}\PY{l+s}{\PYZdq{}}\PY{l+s}{\PYZdq{}}\PY{p}{)}
         
         gsub\PYZus{}func \PY{o}{\PYZlt{}\PYZhy{}} \PY{k+kr}{function}\PY{p}{(}pattern\PY{p}{,} replacement\PY{p}{,} x\PY{p}{)} \PY{p}{\PYZob{}}
           \PY{k+kr}{for}\PY{p}{(}i \PY{k+kr}{in} \PY{l+m}{1}\PY{o}{:}\PY{k+kp}{length}\PY{p}{(}pattern\PY{p}{)}\PY{p}{)}
             x \PY{o}{\PYZlt{}\PYZhy{}} \PY{k+kp}{gsub}\PY{p}{(}pattern\PY{p}{[}i\PY{p}{]}\PY{p}{,} replacement\PY{p}{[}i\PY{p}{]}\PY{p}{,} x\PY{p}{)}
           x
         \PY{p}{\PYZcb{}}
         
         category\PYZus{}4\PY{o}{\PYZdl{}}Min\PYZus{}pressure\PYZus{}mbar \PY{o}{\PYZlt{}\PYZhy{}} \PY{k+kp}{as.numeric}\PY{p}{(}
             gsub\PYZus{}func\PY{p}{(}from\PY{p}{,} to\PY{p}{,} category\PYZus{}4\PY{o}{\PYZdl{}}Min\PYZus{}pressure\PYZus{}mbar\PY{p}{)}\PY{p}{)}
         
         \PY{c+c1}{\PYZsh{}  Clean Category 5 data}
         category\PYZus{}5\PY{o}{\PYZdl{}}Dates \PY{o}{\PYZlt{}\PYZhy{}} \PY{k+kp}{gsub}\PY{p}{(}
             \PY{l+s}{\PYZdq{}}\PY{l+s}{†\PYZdq{}}\PY{p}{,} \PY{l+s}{\PYZdq{}}\PY{l+s}{\PYZdq{}}\PY{p}{,} category\PYZus{}5\PY{o}{\PYZdl{}}Dates\PY{p}{)}
         category\PYZus{}5\PY{o}{\PYZdl{}}Year \PY{o}{\PYZlt{}\PYZhy{}} \PY{k+kp}{as.numeric}\PY{p}{(}
             \PY{k+kp}{gsub}\PY{p}{(}\PY{l+s}{\PYZdq{}}\PY{l+s}{.+, \PYZdq{}}\PY{p}{,} \PY{l+s}{\PYZdq{}}\PY{l+s}{\PYZdq{}}\PY{p}{,} category\PYZus{}5\PY{o}{\PYZdl{}}Dates\PY{p}{)}\PY{p}{)}
         category\PYZus{}5\PY{o}{\PYZdl{}}Month \PY{o}{\PYZlt{}\PYZhy{}} \PY{k+kp}{gsub}\PY{p}{(}
             \PY{l+s}{\PYZdq{}}\PY{l+s}{[0\PYZhy{}9, ].+\PYZdq{}}\PY{p}{,} \PY{l+s}{\PYZdq{}}\PY{l+s}{\PYZdq{}}\PY{p}{,} category\PYZus{}5\PY{o}{\PYZdl{}}Dates\PY{p}{)}
         category\PYZus{}5\PY{o}{\PYZdl{}}DamageUSDMillions \PY{o}{\PYZlt{}\PYZhy{}} \PY{k+kp}{gsub}\PY{p}{(}
             \PY{l+s}{\PYZdq{}}\PY{l+s}{\PYZgt{}\PYZdq{}}\PY{p}{,} \PY{l+s}{\PYZdq{}}\PY{l+s}{\PYZdq{}}\PY{p}{,} category\PYZus{}5\PY{o}{\PYZdl{}}DamageUSDMillions\PY{p}{)}
         category\PYZus{}5\PY{o}{\PYZdl{}}WindSpeedsMPH \PY{o}{\PYZlt{}\PYZhy{}} \PY{k+kp}{as.numeric}\PY{p}{(}
             \PY{k+kp}{gsub}\PY{p}{(}\PY{l+s}{\PYZdq{}}\PY{l+s}{([0\PYZhy{}9]+).*\PYZdq{}}\PY{p}{,} \PY{l+s}{\PYZdq{}}\PY{l+s}{\PYZbs{}\PYZbs{}1\PYZdq{}}\PY{p}{,} category\PYZus{}5\PY{o}{\PYZdl{}}WindSpeedsMPH\PY{p}{)}\PY{p}{)}
         category\PYZus{}5\PY{o}{\PYZdl{}}PressurehPA \PY{o}{\PYZlt{}\PYZhy{}} \PY{k+kp}{as.numeric}\PY{p}{(}
             \PY{k+kp}{gsub}\PY{p}{(}\PY{l+s}{\PYZdq{}}\PY{l+s}{([0\PYZhy{}9]+).*\PYZdq{}}\PY{p}{,} \PY{l+s}{\PYZdq{}}\PY{l+s}{\PYZbs{}\PYZbs{}1\PYZdq{}}\PY{p}{,} category\PYZus{}5\PY{o}{\PYZdl{}}PressurehPA\PY{p}{)}\PY{p}{)}
         
         category\PYZus{}5\PY{o}{\PYZdl{}}DamageUSDMillions \PY{o}{\PYZlt{}\PYZhy{}} \PY{k+kp}{gsub}\PY{p}{(}
             \PY{l+s}{\PYZdq{}}\PY{l+s}{([0\PYZhy{}9]+).*\PYZdq{}}\PY{p}{,} \PY{l+s}{\PYZdq{}}\PY{l+s}{\PYZbs{}\PYZbs{}1\PYZdq{}}\PY{p}{,} category\PYZus{}5\PY{o}{\PYZdl{}}DamageUSDMillions\PY{p}{)}
         category\PYZus{}5\PY{o}{\PYZdl{}}DamageUSDMillions \PY{o}{\PYZlt{}\PYZhy{}} \PY{k+kp}{gsub}\PY{p}{(}
             \PY{l+s}{\PYZdq{}}\PY{l+s}{Extensive\PYZdq{}}\PY{p}{,} \PY{l+s}{\PYZdq{}}\PY{l+s}{\PYZdq{}}\PY{p}{,} category\PYZus{}5\PY{o}{\PYZdl{}}DamageUSDMillions\PY{p}{)}
         category\PYZus{}5\PY{o}{\PYZdl{}}DamageUSDMillions \PY{o}{\PYZlt{}\PYZhy{}} \PY{k+kp}{as.numeric}\PY{p}{(}
             \PY{k+kp}{gsub}\PY{p}{(}\PY{l+s}{\PYZdq{}}\PY{l+s}{\PYZbs{}\PYZbs{}\PYZdl{}\PYZdq{}}\PY{p}{,} \PY{l+s}{\PYZdq{}}\PY{l+s}{\PYZdq{}}\PY{p}{,} category\PYZus{}5\PY{o}{\PYZdl{}}DamageUSDMillions\PY{p}{)}\PY{p}{)}
         
         \PY{k+kp}{head}\PY{p}{(}category\PYZus{}4\PY{p}{)}
         \PY{k+kp}{head}\PY{p}{(}category\PYZus{}5\PY{p}{)}
\end{Verbatim}


    \begin{tabular}{r|lllllll}
 Name & Season & Month & Max\_wind\_knots & Max\_wind\_kmh & Max\_wind\_mph & Min\_pressure\_mbar\\
\hline
	 Hurricane \#3                 & 1853                           & August                         & 130                            & 240                            & 150                            & 924                           \\
	 "1856 Last Island Hurricane" & 1856                         & August                       & 130                          & 240                          & 150                          & 934                         \\
	 Hurricane \#6                 & 1866                           & September                      & 120                            & 220                            & 140                            & 938                           \\
	 Hurricane \#7                 & 1878                           & September                      & 120                            & 220                            & 140                            & 938                           \\
	 Hurricane \#2                 & 1880                           & August                         & 130                            & 240                            & 150                            & 931                           \\
	 Hurricane \#8                 & 1880                           & September                      & 120                            & 220                            & 140                            & 928                           \\
\end{tabular}


    
    \begin{tabular}{r|llllllllll}
 Name & Dates & Duration\_hours & WindSpeedsMPH & PressurehPA & Affected\_Areas & Deaths & DamageUSDMillions & Year & Month\\
\hline
	 "Cuba"                                                                & October 19, 1924                                                      & 12                                                                    & 165                                                                   & 910                                                                   & Central America, Mexico, CubaFlorida, The Bahamas                     &   90                                                                  &  NA                                                                   & 1924                                                                  & October                                                              \\
	 "San Felipe IIOkeechobee"                                             & September 13–14, 1928                                                 & 12                                                                    & 160                                                                   & 929                                                                   & Lesser Antilles, The BahamasUnited States East Coast, Atlantic Canada & 4000                                                                  & 100                                                                   & 1928                                                                  & September                                                            \\
	 "Bahamas"                                                             & September 5–6, 1932                                                   & 24                                                                    & 160                                                                   & 921                                                                   & The Bahamas, Northeastern United States                               &   16                                                                  &  NA                                                                   & 1932                                                                  & September                                                            \\
	 "Cuba"                                                                & November 5–8, 1932                                                    & 78                                                                    & 175                                                                   & 915                                                                   & Lesser Antilles, Jamaica, Cayman IslandsCuba, The Bahamas, Bermuda    & 3103                                                                  &  40                                                                   & 1932                                                                  & November                                                             \\
	 "Cuba–Brownsville"                                                    & August 30, 1933                                                       & 12                                                                    & 160                                                                   & 930                                                                   & The Bahamas, Cuba, FloridaTexas, Tamaulipas                           &  179                                                                  &  27                                                                   & 1933                                                                  & August                                                               \\
	 "Tampico"                                                             & September 21, 1933                                                    & 12                                                                    & 160                                                                   & 929                                                                   & Jamaica, Yucatán Peninsula                                            &  184                                                                  &   5                                                                   & 1933                                                                  & September                                                            \\
\end{tabular}


    
    After cleaning the data, we can try drawing correlation between
integer/numerical type columns. For category 4 data, 3 of the 4
numerical columns are of speed(in different notations), so it is obvious
that they will be highly correlated (\textasciitilde{}1). The important
correlation is between speed and pressure. In our findings, there is a
negative correlation between them, which can be justified according to
http://ww2010.atmos.uiuc.edu/(Gh)/guides/mtr/hurr/stages/cane/pswd.rxml
.

    \begin{Verbatim}[commandchars=\\\{\}]
{\color{incolor}In [{\color{incolor}25}]:} \PY{k+kn}{library}\PY{p}{(}reshape2\PY{p}{)}
         cormat \PY{o}{\PYZlt{}\PYZhy{}} cor\PY{p}{(}category\PYZus{}4\PY{p}{[}\PY{p}{,} \PY{k+kt}{c}\PY{p}{(}\PY{l+s}{\PYZdq{}}\PY{l+s}{Max\PYZus{}wind\PYZus{}knots\PYZdq{}}\PY{p}{,} \PY{l+s}{\PYZdq{}}\PY{l+s}{Max\PYZus{}wind\PYZus{}kmh\PYZdq{}}\PY{p}{,} 
                                      \PY{l+s}{\PYZdq{}}\PY{l+s}{Max\PYZus{}wind\PYZus{}mph\PYZdq{}}\PY{p}{,} \PY{l+s}{\PYZdq{}}\PY{l+s}{Min\PYZus{}pressure\PYZus{}mbar\PYZdq{}}\PY{p}{)}\PY{p}{]}\PY{p}{,} 
                       use \PY{o}{=} \PY{l+s}{\PYZdq{}}\PY{l+s}{complete.obs\PYZdq{}}\PY{p}{,} method \PY{o}{=} \PY{l+s}{\PYZdq{}}\PY{l+s}{pearson\PYZdq{}}\PY{p}{)}
         melted\PYZus{}cormat \PY{o}{\PYZlt{}\PYZhy{}} melt\PY{p}{(}cormat\PY{p}{)}
         
         ggplot\PY{p}{(}data \PY{o}{=} melted\PYZus{}cormat\PY{p}{,} aes\PY{p}{(}x\PY{o}{=}Var1\PY{p}{,} y\PY{o}{=}Var2\PY{p}{,} fill\PY{o}{=}value\PY{p}{)}\PY{p}{)} \PY{o}{+} geom\PYZus{}tile\PY{p}{(}\PY{p}{)}
\end{Verbatim}


    \begin{Verbatim}[commandchars=\\\{\}]

Attaching package: ‘reshape2’

The following object is masked from ‘package:tidyr’:

    smiths


    \end{Verbatim}

    
    
    \begin{center}
    \adjustimage{max size={0.9\linewidth}{0.9\paperheight}}{output_46_2.png}
    \end{center}
    { \hspace*{\fill} \\}
    
    For category 5, below is the correlation matrix. Here also, wind speed
and pressure are negatively correlated.

    \begin{Verbatim}[commandchars=\\\{\}]
{\color{incolor}In [{\color{incolor} }]:} cormat5 \PY{o}{\PYZlt{}\PYZhy{}} cor\PY{p}{(}category\PYZus{}5\PY{p}{[}\PY{p}{,} \PY{k+kt}{c}\PY{p}{(}\PY{l+s}{\PYZdq{}}\PY{l+s}{Duration\PYZus{}hours\PYZdq{}}\PY{p}{,} \PY{l+s}{\PYZdq{}}\PY{l+s}{WindSpeedsMPH\PYZdq{}}\PY{p}{,} 
                                      \PY{l+s}{\PYZdq{}}\PY{l+s}{PressurehPA\PYZdq{}}\PY{p}{,} \PY{l+s}{\PYZdq{}}\PY{l+s}{Deaths\PYZdq{}}\PY{p}{,} \PY{l+s}{\PYZdq{}}\PY{l+s}{DamageUSDMillions\PYZdq{}}\PY{p}{)}\PY{p}{]}\PY{p}{,} 
                       use \PY{o}{=} \PY{l+s}{\PYZdq{}}\PY{l+s}{complete.obs\PYZdq{}}\PY{p}{,} method \PY{o}{=} \PY{l+s}{\PYZdq{}}\PY{l+s}{pearson\PYZdq{}}\PY{p}{)}
        melted\PYZus{}cormat5 \PY{o}{\PYZlt{}\PYZhy{}} melt\PY{p}{(}cormat5\PY{p}{)}
        
        ggplot\PY{p}{(}data \PY{o}{=} melted\PYZus{}cormat5\PY{p}{,} aes\PY{p}{(}x\PY{o}{=}Var1\PY{p}{,} y\PY{o}{=}Var2\PY{p}{,} fill\PY{o}{=}value\PY{p}{)}\PY{p}{)} \PY{o}{+} geom\PYZus{}tile\PY{p}{(}\PY{p}{)}
\end{Verbatim}


    
    

    % Add a bibliography block to the postdoc
    
    
    
    \end{document}
